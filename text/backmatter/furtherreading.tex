\chapter*{Further Reading \\ \large Whereto Now?}
  \markboth{Further Reading}{Further Reading}
  \addcontentsline{toc}{chapter}{Further Reading}
  
If this book has fuelled your desire to learn not only more about the Apocalypse of John and the Bible in general, but also the Ancient Greek language, I believe the following section might be of interest to you. For I have, over the course of my studying Ancient Greek, made good use of a rather large repertoire of various resources and would like to showcase those I believe to be most helpful. 

First and foremost, I highly recommend JACT's ``Reading Greek'' series for commencing your study of the Ancient Greek language. In this series, as opposed to more traditional grammar books, you learn the language by reading as much as possible as soon as possible — as, indeed, the name should have revealed. If you do decide to get yourself a copy of the \textit{Reading Greek} series, I also highly recommend the Italian edition of ``Athenaze'', as it contains a very large amount of beginner-friendly prose. I cannot, however, recommend \textit{Athenaze} as one's only method of learning the language, as it does contain a not insignificant amount of Italian — thus, unless you speak Italian reasonably well, I can only recommend this book as a pairing to the aforementioned \textit{Reading Greek} series.

If you wish to continue reading the New Testament in its original language, my personal favourite is “The Greek New Testament: A Reader's Edition”. It contains not only the entirety of the Greek New Testament and an Greek-English dictionary at the back, but also parsed vocabulary at the bottom of the page; that way, the reader is required to learn only those words that occur thirty times or more in the New Testament — the remaining ones can be found at the bottom of the page. Additionally, all vocabulary is parsed, so that the reader can immediately identify what the conjugation of a particular verb or irregular noun is. Its ISBN is 978-3-438-05168-4.

There also exists a reader’s edition of the Greek Old Testament — which is also known as the “Septuagint(a)” — in a similar style as the above-mentioned New Testament reader; it is called “Septuaginta: A Reader's Edition”. In this rather extensive work — comprised not of one, but two volumes with over 1000 pages each —, you find the entirety of the Old Testament — including a handful of apocrypha —, a Greek-English dictionary and parsed vocabulary at the bottom of each page. It is a rather hefty investment, but the exceptional quality makes it, in my opinion, worthwhile. Its ISBN is 978-3-438-05190-5.

But there exist not only books that may aid you in your journey of studying the Greek Bible — and Greek in general —, but there also exists a rather considerable repository of resources that you can find on the Web. My own website, for example, \url{ancient-greek.net} contains a lot of information on Ancient Greek, how to study it and lots of reviews of various resources I use for learning the language. I highly recommend taking a look at it, as you can not only find aforesaid information thereon, but also links to a myriad of other helpful sites. 
