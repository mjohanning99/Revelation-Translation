\pstart
Ἀποκάλυψις Ἰησοῦ Χριστοῦ, ἣν ἔδωκεν αὐτῷ ὁ Θεός δεῖξαι τοῖς δούλοις αὐτοῦ, ἃ δεῖ γενέσθαι ἐν τάχει, καὶ ἐσήμανεν ἀποστείλας διὰ τοῦ ἀγγέλου αὐτοῦ τῷ δούλῳ αὐτοῦ Ἰωάννῃ, 2ὃς ἐμαρτύρησε τὸν λόγον τοῦ Θεοῦ καὶ τὴν μαρτυρίαν Ἰησοῦ Χριστοῦ, ὅσα τε εἶδε. 3μακάριος ὁ ἀναγινώσκων, καὶ οἱ ἀκούοντες τοὺς λόγους τῆς προφητείας καὶ τηροῦντες τὰ ἐν αὐτῇ γεγραμμένα· ὁ γὰρ καιρὸς ἐγγύς.
\pend

\pstart
4Ἰωάννης ταῖς ἑπτὰ ἐκκλησίαις ταῖς ἐν τῇ Ἀσίᾳ· χάρις ὑμῖν καὶ εἰρήνη ἀπὸ τοῦ ὁ ὢν καὶ ὁ ἦν καὶ ὁ ἐρχόμενος· καὶ ἀπὸ τῶν ἑπτὰ πνευμάτων ἅ ἐστιν ἐνώπιον τοῦ θρόνου αὐτοῦ· 5καὶ ἀπὸ Ἰησοῦ Χριστοῦ, ὁ μάρτυς ὁ πιστός, ὁ πρωτότοκος ἐκ τῶν νεκρῶν, καὶ ὁ ἄρχων τῶν βασιλέων τῆς γῆς. τῷ ἀγαπήσαντι ἡμᾶς, καὶ λούσαντι ἡμᾶς ἀπὸ τῶν ἁμαρτιῶν ἡμῶν ἐν τῷ αἵματι αὐτοῦ· 6καὶ ἐποίησεν ἡμᾶς βασιλεῖς καὶ ἱερεῖς τῷ Θεῷ καὶ πατρὶ αὐτοῦ· αὐτῷ ἡ δόξα καὶ τὸ κράτος εἰς τοὺς αἰῶνας τῶν αἰώνων· ἀμήν. 7ἰδού, ἔρχεται μετὰ τῶν νεφελῶν, καὶ ὄψεται αὐτὸν πᾶς ὀφθαλμός, καὶ οἵτινες αὐτὸν ἐξεκέντησαν· καὶ κόψονται ἐπ’ αὐτὸν πᾶσαι αἱ φυλαὶ τῆς γῆς. ναί, ἀμήν.
\pend

\pstart
8Ἐγώ εἰμι τὸ Α καὶ τὸ Ω, ἀρχὴ καὶ τέλος, λέγει Κύριος, ὁ ὢν καὶ ὁ ἦν καὶ ὁ ἐρχόμενος, ὁ παντοκράτωρ.
\pend

\pstart
9Ἐγὼ Ἰωάννης, ὁ καὶ ἀδελφὸς ὑμῶν καὶ συγκοινωνὸς ἐν τῇ θλίψει καὶ ἐν τῇ βασιλείᾳ καὶ ὑπομονῇ ἐν Ἰησοῦ Χριστοῦ, ἐγενόμην ἐν τῇ νήσῳ τῇ καλουμένῃ Πάτμῳ, διὰ τὸν λόγον τοῦ Θεοῦ καὶ διὰ τὴν μαρτυρίαν Ἰησοῦ Χριστοῦ. 10 ἐγενόμην ἐν Πνεύματι ἐν τῇ Κυριακῇ ἡμέρᾳ· καὶ ἤκουσα φωνὴν ὀπίσω μου μεγάλην ὡς σάλπιγγος, 11λεγούσης, Ἐγώ εἰμι τὸ Α καὶ τὸ Ω, ὁ πρῶτος καὶ ο ἔσχατος· καί, Ὃ βλέπεις γράψον εἰς βιβλίον, καὶ πέμψον ταῖς ἑπτὰ ἐκκλησίαις ταῖς ἐν Ἀσίᾳ, εἰς Ἔφεσον, καὶ εἰς Σμύρναν, καὶ εἰς Πέργαμον, καὶ εἰς Θυάτειρα, καὶ εἰς Σάρδεις, καὶ εἰς Φιλαδέλφειαν, καὶ εἰς Λαοδίκειαν. 12καὶ ἐπέστρεψα βλέπειν τὴν φωνὴν ἥτις ἐλάλησε μετ’ ἐμοῦ· καὶ ἐπιστρέψας εἶδον ἑπτὰ λυχνίας χρυσᾶς, 13καὶ ἐν μέσῳ τῶν ἑπτὰ λυχνιῶν ὅμοιον υἱῷ ἀνθρώπου, ἐνδεδυμένον ποδήρη, καὶ περιεζωσμένον πρὸς τοῖς μαστοῖς ζώνην χρυσῆν· 14ἡ δὲ κεφαλὴ αὐτοῦ καὶ αἱ τρίχες λευκαὶ ὡσεὶ ἔριον λευκόν, ὡς χιών· καὶ οἱ ὀφθαλμοὶ αὐτοῦ ὡς φλὸξ πυρός· 15καὶ οἱ πόδες αὐτοῦ ὅμοιοι χαλκολιβάνῳ, ὡς ἐν καμίνῳ πεπυρωμένοι· καὶ ἡ φωνὴ αὐτοῦ ὡς φωνὴ ὑδάτων πολλῶν. 16καὶ ἔχων ἐν τῇ δεξιᾷ αὐτοῦ χειρὶ αὐτοῦ ἀστέρας ἑπτά· καὶ ἐκ τοῦ στόματος αὐτοῦ ῥομφαία δίστομος ὀξεῖα ἐκπορευομένη· καὶ ἡ ὄψις αὐτοῦ, ὡς ὁ ἥλιος φαίνει ἐν τῇ δυνάμει αὐτοῦ. 17καὶ ὅτε εἶδον αὐτόν, ἔπεσα πρὸς τοὺς πόδας αὐτοῦ ὡς νεκρός· καὶ ἐπέθηκε τὴν δεξιὰν αὐτοῦ χεῖρα ἐπ’ ἐμέ, λέγων μοι, Μὴ φοβοῦ· ἐγώ εἰμι ὁ πρῶτος καὶ ὁ ἔσχατος, 18καὶ ὁ ζῶν, καὶ ἐγενόμην νεκρός, καὶ ἰδού, ζῶν εἰμὶ εἰς τοὺς αἰῶνας τῶν αἰώνων, ἀμήν· καὶ ἔχω τὰς κλεῖς τοῦ ᾅδου καὶ τοῦ θανάτου. 19γράψον ἃ εἶδες, καὶ ἅ εἰσι, καὶ ἃ μέλλει γίνεσθαι μετὰ ταῦτα· 20 τὸ μυστήριον τῶν ἑπτὰ ἀστέρων ὧν εἶδες ἐπὶ τῆς δεξιᾶς μου, καὶ τὰς ἑπτὰ λυχνίας τὰς χρυσᾶς. οἱ ἑπτὰ ἀστέρες ἄγγελοι τῶν ἑπτὰ ἐκκλησιῶν εἰσι· καὶ αἱ ἑπτὰ λυχνίαι ἂς εἶδες αἱ ἑπτὰ ἐκκλησίαι εἰσί.
\pend